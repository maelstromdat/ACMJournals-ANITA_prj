\subsection{Vision and Scope}

Techopedia\footnote{\url{https://www.techopedia.com/definition/2387/cybercrime}} defines Cybercrime as ``[...] a crime in which a computer is the object of the crime (hacking, phishing, spamming) or is used as a tool to commit an offense (child pornography, hate crimes). Cybercriminals may use computer technology to access personal information, business trade secrets or use the internet for exploitative or malicious purposes. Criminals can also use computers for communication and document or data storage. Criminals who perform these illegal activities are often referred to as hackers". As a large-scale phenomenon, Cybercrime hit the headlines in 2017, with the likes of \emph{WannaCry} crippling the National Health Service in May that year, or the Petya/NotPetya ransomware attack infecting global companies shortly thereafter, with a whole host of data breaches from big companies like Equifax. The year 2018 suffered no better fate, indeed \cite{SolanoP17}. In total, the figures around the phenomenon~\footnote{as reported by the InformationAge online media conglomerate.} are staggering: (a) The global cost of Cyber Crime is estimated to reach \$2 trillion by 2019, a threefold increase from the 2015 estimate of \$500 billion; (2) The cost per record stolen averages \$158; (3) in 2018 there were 38\% more cyber-incidents than the previous year; (4) 48\% of crimes are caused with malicious intent --- human error or system failure account for the rest.
%https://www.zdnet.com/article/why-microsoft-is-fighting-to-stop-a-cyber-world-war/?ftag=TRE7ce1dc9&bhid=21378160878801111613213596198479

At the same time, \emph{threat intelligence} is the discipline whose intent is that of providing organized, analyzed, and refined information about potential or current attacks that threaten an organization, including governments, non-governmental organizations, and more \cite{TounsiR18}.

In this paper we aim at providing a synthesis of the state of the art in \emph{cybercrime} threat intelligence, accounting for both grey and white literature on the matter with a systematic \emph{multi-vocal} literature review \cite{slr,garousi2013evaluating}. 

Our results provide a clear overview of most if not all the topics, approaches, indicators, risks, fallacies, and pitfalls around the phenomenon. Our end goal is offering such an overview to encourage analysis, synthesis, and avoidance of the risks connected to cybercrime, starting from techniques existing in the state of the art. The impact of these results is considerable for both practitioners and academicians. On one hand, practitioners may benefit from our results in that they offer indicators, techniques, and tools that help avoiding the damage connected to the phenomenon; on the other hand, the synthesis offered in the following pages offers a starting basis to study the cybercrime phenomenon in deeper detail.

\subsection{Approach and Major Contributions}
\begin{itemize}
    \item overview of research questions with one line of methods to address each one
    \item major contributions outline, with one line to explain the contribution, the intended user and practical impact
\end{itemize}

\subsection{Structure of The Paper}

The rest of this paper is organized as follows. First, Sec. \ref{rel} outlines the background terms and definitions as well as outlining related work. Further on, Sec. \ref{mm} elaborates on the research design behind this study. Subsequently, Sec. \ref{res} outlines the results while Sec. \ref{disc} discusses them in context. Finally, Sec. \ref{conc} concludes the paper\todo{to finish with the paragraphs}.

\section{Background and Related Work}\label{rel}
To the best of our knowledge, this is the first systematic literature review providing a taxonomy about the different type of cybercrime and threat intelligence solutions. However, in the following paragraph, we are discussing some papers that provide a partial overview of threat intelligence rather than cybercrime risks or guidance note in order to assist to address the problem posed by cybercrime. In the online literature there are no surveys trying to create a general overview of the cybersecurity risks and the proposed solutions in order to contain the risks. Our systematic literature review analyses the state of the art of the upcoming cybersecurity risks and the proposed countermeasures today available.
Due to the novelty of the cybersecurity threats and the lack of technologies available to fight the cyber attacks, we will examine also sources from the web like blogs and news, in order to have a broader point of view on the new cybercrime trends.


\subsection{Terms and Definitions}
\todo[fancyline]{we need to add references to each of the elements in the table 1 and also link the table somewhere in the text... else it is dangling! Furthermore, why these terms? we should explain the selection} 


Table~\ref{tab:definitions} lists all the terms and the related definitions used in this study. The table provides on the first column \textit{\textbf{Terms}} the list of those terms considered more technical and more cybersecurity related. On the second column \textit{\textbf{Definitions}} we provide a short explanation of the terms related to the cybersecurity environment. We provide this table in order to help the reader to better understand the whole work, indeed, some of the listed technical words are used in our study, meanwhile other terms could be useful in order to have a better background of the cybersecurity problem we are discussing.


\begin{table}[h!]
\small
%\begin{tabular}{@{}ll@{}}
%\begin{tabular}{ll}
\begin{tabular}{p{0.22\textwidth}p{0.74\textwidth}}
\toprule
\textbf{Terms} & \textbf{Definition} \\ \midrule
\rowcolor[HTML]{EFEFEF} 
Cyber Crime & \begin{tabular}[c]{@{}l@{}} Cybercrime is defined as a crime in which a computer is the object of the crime (hacking, \\phishing, spamming) or is used as a tool to commit an offense (child pornography, hate crimes) \cite{techopedia}.\end{tabular} \\
Surface Web & \begin{tabular}[c]{@{}l@{}}Surface web is the normal web which is visible for all users using internet. The websites in the\\ surface web is indexed by search engines. Google is the great example of search engine.\end{tabular} \\
\rowcolor[HTML]{EFEFEF} 
Deep Web & \begin{tabular}[c]{@{}l@{}}Deep web is the secret web which is not visible for normal user. The deep web consist of\\ a website or any page on the website which are not indexed by search engines.\end{tabular} \\
Dark Web & \begin{tabular}[c]{@{}l@{}}Dark Web is illegal to used. The all criminal activities are act upon on dark web. The criminal\\ activities like drugs dealing, killing humans etc. The user can only access it if the user has\\ Tor Browser.\end{tabular} \\
\rowcolor[HTML]{EFEFEF} 
Threat Intelligence & \begin{tabular}[c]{@{}l@{}}Is an organized, analyzed and refined information about potential or current attacks that threaten\\ an organization. The primary purpose of threat intelligence is helping organizations understand\\ the risks of the most common and severe external threats.\end{tabular} \\
\begin{tabular}[c]{@{}l@{}}Open Source Intelligence\\ (OSINT)\end{tabular} & \begin{tabular}[c]{@{}l@{}}Is the insight gained from processing and analyzing public data sources such as broadcast TV\\ and radio, social media, and websites. These sources provide data in text, video, image, and audio\\ formats.\end{tabular} \\
\rowcolor[HTML]{EFEFEF} 
Crawler & \begin{tabular}[c]{@{}l@{}}A crawler is a program that visits Web sites and reads their pages and other information in order\\ to create entries or retrieve data.\end{tabular} \\
Malware & \begin{tabular}[c]{@{}l@{}}Or ``malicious software'', is an umbrella term that describes any malicious program or code that\\ is harmful to systems. Hostile, intrusive, and intentionally nasty, malware seeks to invade,\\ damage, or disable computers, computer systems, networks, tablets, and mobile devices, often\\ by taking partial control over a device's operations. Like the human flu, it interferes with normal \\functioning.\end{tabular} \\
\rowcolor[HTML]{EFEFEF} 
\begin{tabular}[c]{@{}l@{}}Distributed Denial of Service \\ (DDoS)\end{tabular} & \begin{tabular}[c]{@{}l@{}}A distributed denial-of-service (DDoS) attack is an attack in which multiple compromised\\ computer systems attack a target, such as a server, website or other network resource, and cause\\ a denial of service for users of the targeted resource. The flood of incoming messages, connection\\ requests or malformed packets to the target system forces it to slow down or even crash and \\shut down, thereby denying service to legitimate users or systems.\end{tabular} \\
Watering Hole Attack & \begin{tabular}[c]{@{}l@{}}A watering hole attack is a security exploit in which the attacker seeks to compromise a specific \\group of end users by infecting websites that members of the group are known to visit. The goal\\ is to infect a targeted user's computer and gain access to the network at the target's place of\\ employment.\end{tabular} \\
\rowcolor[HTML]{EFEFEF} 
Spoofing & \begin{tabular}[c]{@{}l@{}}Is a fraudulent or malicious practice in which communication is sent from an unknown source \\disguised as a source known to the receiver. Spoofing is most prevalent in communication\\ mechanisms that lack a high level of security.\end{tabular} \\
Honeypot & \begin{tabular}[c]{@{}l@{}}A honeypot is a decoy computer system for trapping hackers or tracking unconventional or new\\ hacking methods. Honeypots are designed to purposely engage and deceive hackers and identify\\ malicious activities performed over the Internet.\end{tabular} \\
\rowcolor[HTML]{EFEFEF} 
Insider Threat & \begin{tabular}[c]{@{}l@{}}Insider threat is a generic term for a threat to an organization's security or data that comes \\from within. Such threats are usually attributed to employees or former employees, but may \\also arise from third parties, including contractors, temporary workers or customers.\end{tabular} \\
\begin{tabular}[c]{@{}l@{}}Man-in-the-Middle Attack \\ (MITM)\end{tabular} & \begin{tabular}[c]{@{}l@{}}A man-in-the-middle (MITM) attack is a form of eavesdropping where  communication between\\ two users is monitored and modified by an unauthorized party.\end{tabular} \\
\rowcolor[HTML]{EFEFEF} 
Hacktivism & \begin{tabular}[c]{@{}l@{}}Hacktivism is the act of hacking a website or computer network in an effort to convey a social \\or political message. The person who carries out the act of hacktivism is known as a hacktivist.\end{tabular} \\
 \todo[inline]{A} & \todo[inline]{B} \\
\rowcolor[HTML]{EFEFEF}
 A & B \\ \bottomrule
\end{tabular}
\caption{Definition of the terms from the survey paper.}
\label{tab:definitions}
\end{table}



\subsection{Related Surveys}

In a deeply connected world, like the one we are facing nowadays, hackers are constantly finding new targets and refining the tools they use to break through cyberdefenses. Moreover, the lack of privacy and security of the new upcoming technologies and the lack of awareness of the users poses a real threat to our personal life. In the following, we present some works that face the problem of cybersecurity and try to discuss the countermeasures today available. 

Tounsi et al. in \cite{TounsiR18} provide an overview of the open source/free threat intelligence tools and compare their features with those from AlliaCERT TI\footnote{Managed Security Services Division, AlliaCERT Team, Alliacom, France}. Through their analysis, they found that the fast sharing of threat intelligence, as encouraged by any organization in order to cooperate, is not enough to avoid targeted attacks. Moreover, trust is extremely important for companies that are sharing personal information. Another problem is how much data is important to share in order to prevent attacks and cooperate and in which format in order to avoid to lose information. In order to understand which standard is better Tounsi et al. propose their own analysis. Lastly, the work presents a comparison among the best threat intelligence tools dividing them in tools which privilege standardization and automatic analytics and others that focus on high speed requirements. 

Furthermore, If Tounsi et al. focus on what is the best way to keep the trust among organizations and at the same time share information about cyber threats, in Toch et al. \cite{Toch:2018:PIC:3186333.3172869} the authors pose the accent on the type of data required from those cybersecurity systems that are supposed to protect our privacy from prying eyes. The taxonomy suggested in the article shows that almost all cyber-security technological categories require some access to personal sensitive information. This result can offer guidance not only in choosing one technique over another but, more importantly, in designing more privacy-aware cyber-security technologies with little or no compromise with regard to their effectiveness in protecting from cyber attacks. 

The studies from above tried to analyze systems and good practices to mitigate the cyber threats, in Chang et al. \cite{ChangVWL13} we have a study regarding the state-of-the-art of web-based malware attacks and how to defense against. The paper starts with a study of the attack model and the vulnerabilities that enable these attacks, then analyzes the current state of the malware problem, lastly investigates the defense mechanisms. As result, the paper gives three categories of approaches in order to analyze, identify, and defend against the web-based malware problem. Each category with advantages and disadvantages and how these approaches complement each other and how they can work together. 

An altogether different approach from the previous ones is presented in Xu et al. \cite{Xu:2013:CDM:2435349.2435366} where the authors analyze network-layer traffic and application-layer websites contents simultaneously in order to detect the malicious web applications at run-time. The currently available approach to detect malicious websites can be classified into two categories: \textit{static approach} and \textit{dynamic approach}. The first approach analyzes URLs and contents the latter uses clients honeypots to analyze run-time behaviours. The results of this approach showed that cross-layer detection can achieve the same detection effectiveness of the dynamic approach, however it resulted to be much more faster than the dynamic one. 

In order to understand the rising concern around the cybersecurity problem another important reference is also the \emph{Guidance Note}\footnote{Link: \href{}{https://bit.ly/2BIy0tP}} of the United Nations Office on Drugs and Crime (UNODC) that is a global leader in the fight against illicit drugs and international crime. The guidance note aims at giving an overview about the most common cyber security threats today's available. Cybercrimes activities like the online radicalization, or the illicit sales of pharmaceutical solutions rather then frauds and identity theft are presented and explained in order to outline how UNODC can deliver technical assistance in order to address the problems posed by cybercrime at both regional and national levels.
If from scientific literature side we are seeing a huge growing of interest around cybersecurity threats, on the web side we have a lot of blogs and web-pages warning about the new upcoming cybersecurity threats. 

Furtheremore, we refer to reports of one of the major companies working in cybersecurity: Kaspersky\footnote{Link: \href{}{https://www.kaspersky.com}}. On the Kaspersky Threats blog page\footnote{Link: \href{}{https://bit.ly/2AijYiF}} where the company offers an updated list of the new upcoming cyber threats. More specifically, for example, on the top five worst cybersecurity attacks we have WannaCry and NotPetya/ExPetr two famous ransomware encryptor that use to encrypt the data of the victim user. Stuxnet a worm that targets the types of industrial control systems (ICS) that are commonly used in infrastructure supporting facilities (i.e. power plants, water treatment facilities, gas lines, etc). DarkHotel a spyware in order to conduct targeted phishing attacks using the hotel's Wi-Fi networks. In addition, Mirai is a botnet used to flood the DNS service provider Dyn with requests. The Kaspersky company gives guidelines \cite{kaspersky} on how address incident response in order to contain a cybersecurity attack. Kaspersky listed some key-points necessarily for a company to avoid and contain attacks: \emph{(i)} the speed a rapid remediation is key to limiting the costs, \emph{(ii)} proactive protection, \emph{(iii)} presence Of internal specialists. However, in order to have an world wide overview about real time cyber attacks, Kaspersky provided the Cyberthreat Real-Time Map available here \href{}{https://cybermap.kaspersky.com/} where is possible to see the current cyber attacks around the globe. 

Althogether, however, although plenty of white/grey literature exists on the topic, a holistic view over what software, indicators, methods, tools, and approaches to cyber-crime fighting that practitioners and law-enforces can use is still nowhere to be seen. We offer an initial attempt at such a review in the coming pages, for the benefit of practitioners and academicians alike.

%Anoter relevant related work in terms of cyber-threat intelligence is also the grey literature in the MIT series \cite{MIT} from M. Giles, a blog-site where the author lists and explains six major cybersecurity threats. \emph{Data breaches}, companies store a lot of personal and sensitive data, Marc Goodman, a security expert, thinks that people's personal web browsing habits will be the next popular targets. \emph{Ransomware in the cloud}, one big target in 2018 will be cloud computing businesses, which store a lot of data for companies. Subsequently, \emph{The weaponization of AI} entails AI driven arms to be supported by machine-learning models in order to better anticipate attacks. Hackers could take advantage of this to drive more phishing attacks. \emph{Cyber-physical attacks}, the new attacks could involve electrical grids, transportation systems, and other parts of countries' critical infrastructure. \emph{Mining cryptocurrencies} the act of mining needs a vast amounts of computing capacity to solve complex mathematical problems. This could encourage hackers to compromise computers in order to use them. \emph{Hacking election} fake news and cyberattacks on the voting process could be the next (?) threat we will face of. 
Another major company working in cybersecurity is Norton\footnote{Link: \href{}{https://us.norton.com/}}. In \cite{norton} a Symanthec employee gives a pictures of cybersecurity threats and the impact they have on the American population. Mobile malware and Third-party app stores seem to be the new concern. Spyware, ransomware, and viruses used to focus on laptop or desktop computer, however since 2017 the malware variants for mobile increased 54 percent. As well, Symantec, found third-party app stores hosted 99.9 percent of discovered mobile malware. From the Symantec report we can read that in 2023 cybercriminals will steal an estimated 33 billion records that might include your name, address, credit card information, or Social Security number. The impact of this identity theft will impact 60 million Americans and the average costs have been estimated in \$3.86 million (U.S. dollars) for the companies worldwide and \$7.91 million (U.S. dollars) for the U.S. company.
Our survey study try to create an overview among cyberthreats providing a taxonomy of the current criminal activities and complementary we provide on overview of indicators and risks parameters in order to detect cyber crime activities. To the best of our knowledge this is the first systematic literature review on cybersecurity and threat intelligence.

