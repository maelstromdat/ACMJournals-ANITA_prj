%\begin{enumerate}
%\item recap of what we did, tailor and modify from the intro
%\item address each research question with each of the results and pointers to the relative sections; discuss here the impact and practical uses
%\item elaborate on practical impact, final considerations (what's missing) 
%\item future work?
%\end{enumerate}

This paper provides a Systematic Multi-Vocal Literature Review on the methods, indicators, approaches and techniques previously explored for the purpose of cybercrime threat intelligence, namely, the act of gathering information over, predicting, avoiding, or prosecuting cyber-criminal activities in the surface-, deep-, and dark-webs. More specifically, the attained results provide an overview of the state of the art over (a) what online depth levels are assessed and to what extent; (b) what degrees of anonymity exist for web-crawling; (c) what policies exist to vary the degrees of anonymity; (d) what website features are most indicative of cyberthreats; (e) what risk assessment techniques exist.

Overall, our data, results, and discussions support three conclusions.

First, there is a distinct gap between the grey literature --- which mainly discusses reported vulnerabilities as well as organisational/economical/financial consequences of being targeted by cybercriminal activity --- and the white research literature --- which mainly focuses on offering scattered non-definitive attempts at predicting, avoiding, or protecting against specific criminal-activity types. To address this gap, we discussed our results and the limitations therein, also offering a preliminary formulation of a holistic metric to assess the risk-level that any given online source may be theatre to online criminal activity.

Second, no single community encapsulates cybercrime-fighting software, tools, approaches and techniques, rather, these techniques or their relevant related work is scattered across as many as 30+ domain-specific communities (e.g., software security, data privacy, software engineering, distributed computing, artificial intelligence, and more). In discussing this observation we offered descriptive statistics over our sample in the hope of pointing community leaders in the right direction while fostering cross-fertilisation or community-building.

Third, finally, there is no one definitive solution towards assisting law-enforcement agencies in their cybercrime-fighting activity. A holistic integration effort is advised.



In the future we plan to address the above shortcomings even further, to the extent that, (1) we aim at providing a holistic tool to aid law-enforcers in combatting and prosecuting online criminal activity, (2) we aim at fostering a data-driven, cybercrime-fighting practitioners community and (3) most immediately, we aim at building a tool for large-scale online datasource risk-assessment of criminal activity. We plan to conduct and refine the above activities in the scope of the EU ANITA H2020 project in direct synergy with the law-enforcement practitioners within the ANITA consortium.